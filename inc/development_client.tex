\section{Реализация клиентской части проекта}

\subsection{Архитектура приложения}

Согласно требованиям архитектура приложения должна соответствовать шаблону проектирования MVVM\cite{8}. На рисунке \ref{MVVM} представлена структура нашего приложения. Для изображения диаграммы использована среда проектирования Visual Paradigm Community Edition v.16.2.

\addimghere{MVVM}{1}{Архитектура приложения}{MVVM}

В классе ViewModel будет реализована логика приложения. View отвечает за работу с компонентами пользовательского интерфейса. Для отображения в пользовательском интерфейсе даных из ViewModel применяется хранилище данных LiveData\cite{9}.
LiveData - хранилище данных, работающее по принципу паттерна Observer (наблюдатель). Это хранилище умеет делать две вещи:
\begin{enumerate}
	\item В него можно поместить какой-либо объект;
	\item На него можно подписаться и получать объекты, которые в него помещают.
\end{enumerate}

Activity подписывается на LiveData и получает данные, которые помещает в него ViewModel. В Repository будут реализованы методы для получения данных из нашего сервера. Для описания и отправки HTTP-запросов применяется библиотека Retrofit\cite{10}. Интерфейсы всех необходимых запросов описываются в APIService.


\subsection{Реализация приложения}

Пользовательский интерфейс приложения состоит из 5 страниц(Fragments) наследуемые от одного Activity:
\begin{enumerate}
	\item Главная;
	\item Авторизация;
	\item Регистрация;
	\item Состав продукта;
	\item Список недопустимых продуктов.
\end{enumerate}

На рисунке \ref{NavGraf} представлен разработанный навигационный граф нашего приложения.

\addimghere{NavGraf}{1}{Навигационный граф приложения}{NavGraf}

Рассмотрим подробнее реализованные страницы приложения. На рисунке \ref{home} представлена разработанная домашняя страница клиентского приложения. Данная страница содержит следующие компоненты:
\begin{enumerate}
	\item поле для ввода штрих-кода продукта;
	\item кнопку для сканирования штрих-кода через камеру на телефоне
	\item кнопку получить состав
\end{enumerate}

При активации кнопки <<Получить состав>> отправляется HTTP-запрос на сервер для получения необходимой информации о продукте и переход на страницу "Состав продукта". Для получения успешного ответа от сервера в заголовке запроса необходимо передать токен доступа, для этого пользователь должен быть авторизован. Страница авторизации представлена на рисунке \ref{login}.
\addimghere{home}{0.5}{Домашняя страница}{home}
При условии отсутствия исключении вовремя выполнения запроса, ViewModel получает информацию о продукте от сервера и заносит в хранилище LiveData. Наблюдатель, <<Состав продукта>> представленный на рисунке \ref{compos}, видит изменения в хранилище данных и отображает данные для пользователя. 
\addimghere{compos}{0.5}{Состав продукта}{compos}
При авторизации пользователя ViewModel получает токены доступа и обновления. Для сохранения токенов и передачи в заголовок каждого запроса  используется SharedPreferences\cite{11}. SharedPreferences позволяет  сохранять данные в виде пар ключ-значение. Android хранит SharedPreferences в виде XML файла в папке sharedprefs. Это позволяет исключить потерю токеннов при закрытии приложения и отказаться от использования базы данных на стороне клиента. Использование приватного режима сохранения (PRIVATE), не позволит другим приложениям иметь доступ к этим файлам и обеспечит необходимую безопасность. Для регистрации пользователя на сервере необходимо активировать кнопку <<Регистрация>> и приложение перейдет на необходимую страницу. Пользовательский интерфейс страницы регистрации приведен на рисунке \ref{registration}.


\addimghere{login}{0.55}{Страница авторизации}{login}

\addimghere{registration}{0.55}{Страница регистрации}{registration}

На рисунке \ref{unaccept} представлена страница добавления  недопустимых для пользователя продуктов.

\addimghere{unaccept}{0.5}{Добавление недопустимых продуктов}{unaccept}

На данном этапе разработки приложения реализован только пользовательский интерфейс этой страницы.  Бизнес-логику добавления и удаления недопустимых продуктов планируется реализовать при дальнейшем развитии проекта.










\clearpage