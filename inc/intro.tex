\anonsection{Введение}


Питание является одним из основных факторов, влияющих на здоровье человека. В настоящее время заметно возрастает понимание того, что пища оказывает на человека значительное влияние. Она даёт энергию, силу, развитие, а при грамотном её употреблении – и здоровье. Без всякого сомнения, можно утверждать, что здоровье человека на 65\% зависит от питания. Вредная пища зачастую является основным источником большинства заболеваний. Повышенное содержание холестерина в крови, ожирение, кариес, диабет, нарушение жирового обмена веществ, гипертония, повышенное содержание мочевой кислоты в крови – вот неполный перечень болезней вызванных неправильным питанием. Исключив из своего рациона вредные продукты, можно не только предотвратить эти заболевания, но и избавиться от большинства хронических недугов. С каждым годом увеличивается ассортимент готовых продуктов питания и усложняется их состав. Вместе с тем растут и проблемы выбора здоровой пищи возле прилавков магазина. Ведь не всегда просто разобрать состав продукта, написанный мелким шрифтом, и определить подходит ли данный товар вам. Особенно эта проблема актуальна для людей со слабым зрением или имеющие аллергию на определенные продукты. Для облегчения процесса выбора  продуктов питания, было принято решение, реализовать приложение <<ShoppingAssistant>>(Помощник по покупкам).

	ShoppingAssistant ускоряет и облегчает процесс покупок в любых
продовольственных магазинах. Просто отсканировав штрих-код продукта
с помощью камеры на смартфоне, через приложение, получаешь полную
информацию о составе и пищевой ценности продукта в удобочитаемом
виде. В приложении можно заранее составлять список нежелательных
продуктов в составе товара, которые будут подсвечиваться определенным цветом предупреждая вас. Также ведется подсчет калорий и соотношения белков, жиров и углеводов в корзине для соблюдения баланса энергетической ценности.

Целью данной работы является разработка клиент-серверного приложения <<ShoppingAssistant>>, ориентированное на людей следящих за
своим питанием:
\begin{enumerate}
	\item Людей имеющих аллергию на определенные продукты;
	\item Вегетарианцев;
	\item Людей соблюдающих пост по религиозным или иным причинам;
	\item Людей соблюдающих диету для похудения или для набора мышечной массы.
\end{enumerate}

При релизации сервера придерживаться архитектурного стиля REST(Representational State Transfer — «передача состояния представления»). Для взаимодействия с клиентом сервер должен предоставлять API(Application Programming Interfaces  — «интерфейс прикладного программирования»). Использовать данные полученные и подготовленные в рамках курса <<Нереляционные базы данных>>.
При реализации клиентского приложения использовать архитектурный паттерн MVVM.
\clearpage