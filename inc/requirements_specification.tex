\section{Анализ предметной области}

\subsection{Исследование предметной области}

В ходе исследования выявлены следующие сущности характеризующие предметную область данной работы:
\begin{enumerate}
	\item Пользователь -- потенциальный клиент пользующийся услугами разрабатываемого приложения
	\item Продукт -- переработанный и произведенный продовольственный товар предназначенный для употребления человеком в пищу. 
	\item Недопустимый продукт -- нежелательный к употреблению ингредиент для пользователя, входящий в состав продукта. 
\end{enumerate} 

Целью пользователя является получение информации о продукте. Поэтому пользователю необходимо предоставить удобный инструмент для поиска продуктов в разрабатываемом сервисе. Для этого необходимо однозначно идентифицировать продукты в системе. Одним из возможных вариантов решения этой проблемы, является использование штрих-кода. 
Рассмотрим более подробно процесс маркировки товаров и выясним целесообразность использования штрих-кода в качестве уникального идентификатора продукта. 

Каждый произведенный и допущенный к продаже продовольственный товар должен маркироваться штрих-кодом. Штрих-код это последовательность чёрных и белых полос, нанесенная на поверхность, маркировку или упаковку товаров, предоставляющая возможность считывания её техническими средствами. Двоичная система цифр обеспечивает удобную запись штрих кодов. Штрихи обозначаются цифрой «1», а пробелы «0».
В штрих-код зашифровывается номер GTIN. GTIN обеспечивает однозначную идентификацию товара в любой стране мира и не может быть присвоен никакому иному товару. Функции администрирования номера GTIN в каждой стране возложена на национальную организацию GS1. GTIN имеет четкую структуру — то есть строится по определенным правилам. В зависимости от назначения товара, ему может быть присвоено GTIN длиной 8, 12,13 или 14 цифр. Большинство товаров во всем мире кодируют системой EAN-13. Код EAN-13 состоит из 13 цифр и имеет следующую структуру:
\begin{enumerate}
	\item Первые 3 цифры кода EAN-13 обозначают страну производства товара;
	\item Следующие 4 цифры являются кодом предприятия изготовителя товара;
	\item Следующие 5 цифр – это код товара по классификации изготовителя;
	\item 13-я цифра – контрольное число, которое вычисляется из предыдущих двенадцати.
\end{enumerate} 

Нанесение штрих-кода на товар осуществляет производитель с применением данных о стране местонахождения и кода производителя. Из вышесказанного следует, что штрих-код является лучшим выбором для идентификации продуктов в нашем сервисе. Использование такого подхода позволит реализовать сканер штрих-кода, что несомненно увеличит удобство использования приложения.


\subsection{Выявление функциональных и системных требований}
Реализуемый сервис должен удовлетворять следующим функциональным требованиям:
\begin{enumerate}
	\item Регистрация пользователя. Все данные пользователя должны храниться в БД. Для обеспечения безопасности, пароль доступа должен храниться в зашифрованном виде;
	\item Авторизация пользователя. Процесс аутентификации пользователя должна происходить с помощью токенов доступа. Сервер должен иметь возможность обновления токена;
	\item У каждого зарегистрированного пользователя в базе данных, должен храниться список недопустимых продуктов. Пользователь должен иметь возможность добавлять и удалять необходимые продукты;
	\item По штрих-коду товара сервер должен предоставить пользователю:
	 \begin{itemize}
	 	\item состав товара;
	 	\item пищевую и энергетическую ценность товара;
	 	\item список недопустимых продуктов входящих в состав товара. 
	\end{itemize}
\end{enumerate}

Для реализуемого сервиса предъявляются следующие системные требования:

\begin{enumerate}
	\item При проектировании сервера использовать архитектурный стиль REST;
	\item Для передачи данных клиенту и взаимодействия с базой данных использовать формат JSON;
	\item Для реализации системы авторизации использовать технологию и стандарт JWT(JSON Web Token). На сервере не должна храниться никакая  информация о состоянии клиента;
	\item Импортировать в базу данных разрабатываемого сервера данные, полученные и подготовленные в рамках курса <<Нереляционные базы данных>>.
	\item Клиент должен быть реализован для операционной системы Android c применением шаблона проектирования MVVM. 
\end{enumerate}

\subsection{Анализ существующих аналогов приложения}

В ходе поиска аналогов со схожим функционалом, были найдены два приложения <<Open Food Facts>> и <<Состав продуктов>>. Мы провели небольшой анализ этих приложений, в ходе которого выявили достоинства и недостатки каждого из них.

<<Open Food Facts>> -- это бесплатная онлайн-база данных о пищевых продуктах со всего мира, функционирующая под лицензией Open Database License (ODBL). Проект призван собирать информацию и данные о пищевых продуктах со всего мира. Для каждого продукта в базе данных хранится его общее название, тип упаковки, бренд, категория, места производства, страны и магазины, где продается продукт, список ингредиентов. Для взаимодействия с этой базой данных реализовано приложение, которое можно скачать в Google Play.
\newline
  
Достоинства:
\begin{enumerate}
	\item Большая база данных с продуктами по всему миру.
	\item Возможность просмотра полной информации о продукте
	\item Поиск продукта по штрих-коду
	\item Наличие сканера штрих-кода 
\end{enumerate}

Недостатки:
\begin{enumerate}
	\item В базе данных представлено мало российских продуктов  
	\item Не удобное для чтения отображение информации о продукте
	\item Отсутствует возможность добавления недопустимых для пользователя продуктов в составе товара.
\end{enumerate}

<<Состав продуктов>> -- приложение предоставляет информацию о питательной ценности продуктов и их химическом составе: содержание белков, жиров, углеводов, калорийности, витаминов, минеральных веществ.

Достоинства:
\begin{enumerate}
	\item Большая база данных с российскими продуктами.
	\item Возможность просмотра полной информации о продукте
\end{enumerate}

Недостатки:
\begin{enumerate}
	\item Отсутствует поиск по штрих коду продукта. Присутствует возможность выбора из предложенных категорий или поиска по названию. В отличие от штрих кода название не является уникальным свойством продукта и не может его полностью идентифицировать. 
	\item Отсутствует возможность добавления недопустимых для пользователя продуктов в составе товара.
\end{enumerate}

В ходе проведенного анализа не найдены аналоги покрывающие весь функционал разрабатываемого приложения. 







\clearpage