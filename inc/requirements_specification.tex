\section{Постановка задачи}


Разработать клиент-серверное приложения <<ShoppingAssistant>>.
Приложение <<ShoppingAssistant>> ориентирована на людей следящих за
своим питанием:
\begin{enumerate}
	\item Людей имеющих аллергию на определенные продукты;
	\item Вегетарианцев;
	\item Людей соблюдающих пост по религиозным или иным причинам;
	\item Людей соблюдающих диету для похудения или для набора мышечной массы.
\end{enumerate}


При релизации сервера придерживаться архитектурного стиля REST(Representational State Transfer — «передача состояния представления»). Для взаимодействия с клиентом сервер должен предоставлять API(Application Programming Interfaces  — «интерфейс прикладного программирования»). Использовать данные полученные и подготовленные в рамках курса <<Нереляционные базы данных>>.
При реализации клиентского приложения использовать архитектурный паттерн MVVM.


\subsection{Функциональные требования}
Реализуемый сервис должен удовлетворять следующим функциональным требованиям:
\begin{enumerate}
	\item Регистрация пользователя. Все данные пользователя должны храниться в БД. Для обеспечения безопасности, пароль доступа должен храниться в зашифрованном виде;
	\item Авторизация пользователя. Процесс аутентификации пользователя должна происходить с помощью токенов доступа. Сервер должен иметь возможность обновления токена;
	\item У каждого зарегистрированного пользователя в базе данных, должен храниться список недопустимых продуктов. Пользователь должен иметь возможность добавлять и удалять необходимые продукты;
	\item По штрих-коду товара сервер должен предоставить пользователю:
	 \begin{itemize}
	 	\item состав товара;
	 	\item пищевую и энергетическую ценность товара;
	 	\item список недопустимых продуктов входящих в состав товара. 
	\end{itemize}
\end{enumerate}

\subsection{Системные требования}

Для реализуемого сервиса предъявляются следующие системные требования:

\begin{enumerate}
	\item При проектировании сервера использовать архитектурный стиль REST;
	\item Для передачи данных клиенту и взаимодействия с базой данных использовать формат JSON;
	\item Для реализации системы авторизации использовать технологию и стандарт JWT(JSON Web Token). На сервере не должна храниться никакая  информация о состоянии клиента;
	\item Импортировать в базу данных разрабатываемого сервера данные, полученные и подготовленные в рамках курса <<Нереляционные базы данных>>.
	\item Клиент должен быть реализован для операционной системы Android c применением шаблона проектирования MVVM. 
\end{enumerate}


\clearpage