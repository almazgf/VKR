\section{Реализация серверной части проекта}
Для проектирования системы использовался универсальный язык моделирования UML с использованием редактора диаграмм Visual Paradigm Community Edition v.16.2\cite{3}. При проектировании системы, соответствующая вышеописанным требованиям, были разработаны следующие диаграммы:

\begin{enumerate}
	\item диаграмма вариантов использования платформы;
	\item диаграммы описания ресурсов RestAPI 
\end{enumerate}

\subsection{База данных}
В качестве базы данных было решено использовать NoSQL СУБД MongoDB\cite{4}. Подробное изучение данной СУБД позволило выявить ряд нижеописанных преимуществ для нашей системы:
\begin{enumerate}
	\item Документоориентированность – СУБД предназначена для хранения иерархических структур данных (документов).
	\item Представление в формате JSON – формат данных JSON удобен в интерпретации и использовании. В виду того, что серверная составляющая должна предоставлять RestAPI, MongoDB отлично подходит для реализации нашего сервиса.
	\item Гибкость и динамичность данных – MongoDB способен хранить
	как структурированные, так и не структурированные данные.
	\item Документация и ПО – данная СУБД является ПО с открытым исходным кодом, с отличной документацией и реализацией драйверов для подавляющего большинства языков программирования.
\end{enumerate}

Исходя из системных требований к проектируемому сервису, представление данных в формате JSON стал решающим фактором в выборе данного СУБД. Это позволит исключить использование дополнительных прослоек в виде ORM для преобразования данных.

Контейнерами сущностей (документов) в MongoDB являются коллек-
ции. Ниже описаны используемые коллекции с перечислением ключевых свойств документов.

Продукты
\begin{lstlisting}
product = {
	"name": string,		`-- название продукта`
	"barcode": string,		`-- штрих-код продукта `
	"composition":[array],	`-- состав продукта`
	"comment": string,		`-- комментарии к продукту`
	"gost": string,		`-- номер ГОСТ-стандарта`
	"net mass": string,		`-- масса-нетто продукта`
	"keeping time": string,		`-- срок годности`
	"storage conditions": string,	`-- условия хранения`
	"esl": {  						`-- пищевая ценность продукта`
			"protein": number, `-- количество белков на 100 грамм продукта`
			"fats": number,	`-- количество жиров на 100 грамм продукта`
			"carbohydrates": number, `-- количество углеводов на 100 грамм`
			"calorie": number, `-- количество калорий на 100 грамм продукта`
	}
	"packing type": string  `-- тип упаковки`
}
\end{lstlisting}

Пользователи
\begin{lstlisting}
users = {
	"_id": Objectid,	`-- уникальный идентификатор пользователя` 
	"name": string,		`-- имя пользователя(логин)`			
	"password": string,		`-- хеш пароля пользователя`
	"unacceptable_products": [array]	`-- список недопустимых продуктов`
}
\end{lstlisting}

Черный список токенов обновления
\begin{lstlisting}
token_blacklist = {
	"_id": Objectid,	`-- уникальный идентификатор` 
	"jti": string,		`-- токен обновления`			
	"created_at": date,		`-- время добавления токена в формате UTC`
	"user_id": Objectid		`-- уникальный идентификатор пользователя`
}
\end{lstlisting}

В черном списке будут храниться токены обновления пользователей с неистекшим сроком жизни, чтобы исключить возможность повторного использования для получения токена доступа. По истечении срока жизни токены автоматический удаляются из коллекции.

\subsection{RestAPI}

Для описания общего представления о функциональном назначении
платформы была разработана диаграмма вариантов использования представленная на рисунке \ref{usecase}. С платформой взаимодействует два актера:
\begin{enumerate}
	\item гость -- пользователь не авторизованный в системе;
	\item авторизованный пользователь. 
\end{enumerate}
\addimghere{usecase}{0.80}{Архитектура полной виртуализации}{usecase}

Гость платформы имеет следующие варианты использования:
\begin{enumerate}
	\item регистрация в системе;
	\item  авторизация в системе.
\end{enumerate}

Авторизованный пользователь имеет следующие варианты использования:
\begin{enumerate}
	\item получение продукта -- получение полной информации о продукте по штрих-коду и списка недопустимых продуктов присутствующих в составе;
	\item добавление недопустимых продуктов -- добавление нежелательного продукта в коллекцию;
	\item удаление недопустимых продуктов -- удаление нежелательного продукта из коллекции.
\end{enumerate}

Для дальнейшего проектирования необходимо ввести понятие <<ресурс>>. Ресурс в REST-архитектуре является ключевой абстракцией информации. Любая информация, которая может быть названа, может быть ресурсом.
Сам Рой Филдинг в своей диссертации дает такое определение:<<Любая информация, которая может быть названа, может быть ресурсом>>. Другими словами, любое понятие, которое может быть объектом гипертекстовой ссылки, должно вписываться в определение ресурса. Исходя из требований к реализуемому проекту можно выделить следующие ресурсы:
\begin{enumerate}
	\item пользователи;
	\item продукты;
	\item недопустимые продукты;
	\item токены.
\end{enumerate}

Проектирование будем производить в среде visual paradigm, который предоставляет сервис для моделирования RestAPI. На рисунке \ref{user_resource} представлен модель RestAPI для ресурса пользователь. Для данного ресурса доступны два действия: регистрация и авторизация. На модели белая стрелка в черном квадрате направленная внутрь блока действия, обозначает запрос(request) к серверу, а обратная ответ(response). В заголовке блока действия указывается тип HTTP-запроса и URL. В синих блоках указываются параметры запроса и формат передаваемых параметров. При регистрации пользователя на сервер отправляется POST-запрос с необходимыми параметрами по указанному URL. В зависимости успешности действия возвращается соответствующй ответ. На рисунках \ref{product_resourse} -- \ref{token_resourse} представлены модели для остальных ресурсов. Не будем приводить описание для каждого ресурса, так как они анологичны приведенному выше.
\addimghere{user_resource}{0.99}{Модель RestAPI для ресурса Пользователи}{user_resource}
\addimghere{product_resourse}{0.80}{Модель RestAPI для ресурса Продукты}{product_resourse}
\addimghere{unacceptable_product}{1}{Модель RestAPI для ресурса Недопустимые продукты}{unacceptable_product}
\addimghere{token_resourse}{0.60}{Модель RestAPI для ресурса Токены}{token_resourse}

В данной главе представлено подробное описание компонентов RestAPI. В соответствии с функциональными требованиями к платформе
была спроектирована диаграмма вариантов использования платформы.
Разработаны модели для каждого ресурса, в соответствии с которыми будет реализован будущий сервер.

\subsection{Выбор инструментов разработки}

Исходя из системных требований предъявляемых к реализуемому серверу, к выбору инструментов разработки накладываются следующие ограничения:
\begin{enumerate}
	\item Поддержка формата передачи данных JSON;
	\item Наличие библиотек для работы со стандартом JWT;
	\item Возможность реализации архитектурного стиля REST.
\end{enumerate}

Для реализации сервера используется язык программирования Python версии 3.8.5 и фреймворк для создания веб-приложений
Flask\cite{5}. Flask предоставляет гибкую настройку веб-приложения с минимальным количеством зависимостей. Пакетный менеджер pip предоставляет множество расширений для языка Python, благодаря чему реализация взаимодействия между компонентами системы сводится к написанию бизнес логики. Процесс написания программного кода и его отладки осуществлялся в интегрированной среде разработки PyCharm версии 2020.3.3. Для взаимодействия с базой данных используется рекомендованный способ для работы с MongoDB из среды Python пакет PyMongo\cite{6}. Пример подключения к базе данных:
\begin{lstlisting}
app.config["MONGO_URI"] = "mongodb://localhost:27017/barcode"
mongo = PyMongo(app)
\end{lstlisting}

Создается экземпляр класса клиента базы данных, который используется для обращения к данным. Благодаря тому, что данная база является документоориентированной с представлением объектов в типе BSON отсутствует миграция базы данных.

Для реализации стандарта аутентификации используется расширение
Flask-JWT-Extended\cite{7}, которое реализует весь необходимый функционал. Для встраивания данного функционала в программный код, достаточно добавить к методам ресурсов соответствующие декораторы доступа. При возникновении ошибки декоратор прерывает выполнение обработки запроса и возвращает соответствующее сообщение. По умолчанию access-токен валиден в течение 15 минут, а refresh-токен в течение 30 дней. Для авторизованного запроса необходимо передать параметр Authorization в заголовке запроса с содержимым: Bearer <соответствующий токен доступа (access/refresh)>.


\subsection{Описание реализованного RestAPI}

Описание основных параметров конфигурации реализованного сервера:
\begin{enumerate}
	\item SECRET KEY -- секретный ключ, который использует Flask при
	вычислении хэша паролей. Пакетное расширение jwt extended использует его для генерации подписи при создании токенов. Секретный ключ хранится отдельно от программного кода и передается посредством переменных окружения (environment variables), при развертке данного приложения на сервере;
	\item JWT ACCESS TOKEN EXPIRES -- срок жизни для токена доступа.  Задаем равной 30 минутам;
	\item JWT REFRESH TOKEN EXPIRES -- срок жизни для токена обновления. Задаем равной 24 часам;
	\item MONGO URI -- URL сервера с MongoDB с необходимыми
	данными авторизации;
\end{enumerate}

Опираясь на модели спроектированные в предыдущей главе, были разработаны методы для каждого ресурса требуемые для полноценной работы сервиса, которым присвоены соответствующие маршруты. В таблице \ref{APIdescription} приводится описание всех реализованных методов. Листинг кода реализованного сервера приведен в приложении А.  

\begin{singlespacing}
	\begin{table}[H]
		\caption{Описание реализованного RestAPI}\label{APIdescription}
		\begin{tabular}{|  p{80pt} |  p{80pt} |  p{290pt} |}
			\hline 
			URL & Метод & Описание \\
			\hline
			/registration & POST & Регистрация пользователя с переданными ар-
			гументами. Результатом является сообщение об успешной регистрации пользователя, либо сообщеие о соответствующей ошибке с HTTP-кодом. \\
			\hline
			/login & POST & Аутентификация пользователя по логину и па-
			ролю. Результатом является access- и refresh- токены, либо сообщеие о соответствующей ошибке с HTTP-кодом. \\
			\hline
			/refresh & POST & Обновление access-токена на основе передан-
			ного refresh-токена. Результатом является access-токен. \\
			\hline
			/unacceptable products & POST & Добавление недопустимого продукта для пользователя. Необходим заголовок авторизации. Результатом является сообщение об успешном добавлении продукта, либо сообщеие о соответствующей ошибке с HTTP-кодом. \\
			\hline
			/unacceptable products & DELETE & Удаление недопустимого продукта из списка хранимой в базе данных. Необходим заголовок авторизации. Результатом является сообщение об успешном удалении продукта, либо сообщеие о соответствующей ошибке с HTTP-кодом. \\
			\hline
			/product & POST & Получение информации о продукте по преданному штрих-коду. Необходим заголовок авторизации. Результатом является полнаяинформация о продукте, либо сообщеие о соответствующей ошибке с HTTP-кодом. \\
			\hline
		\end{tabular}
		Источник: собственная разработка
	\end{table}
\end{singlespacing}
Все ресурсы на стороне сервера требуют авторизации с использованием токенов доступа, за исключением ресурсов, необходимых для аутентификации. Refresh-токен позволяет клиентам запрашивать новые access-токены по истечении их времени жизни. Refresh-токен выдается на более длительный срок, чем access-токен, и по истечению времени его жизни клиенту необходимо вновь пройти процесс аутентификации. 



\clearpage